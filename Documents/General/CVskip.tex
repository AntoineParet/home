\documentclass[10pt,a4paper,sans]{moderncv}
\moderncvtheme[blue]{classic}
\moderncvcolor{blue}
\moderncvstyle{classic}

%Gummi|065|=)

\usepackage{graphicx}
\usepackage[utf8]{inputenc}
\usepackage[T1]{fontenc}
\usepackage[french]{babel}
\usepackage[cyr]{aeguill}
\usepackage[top=1.5cm, bottom=1.5cm, left=3cm, right=3cm]{geometry}
\setlength{\hintscolumnwidth}{2.5cm}


\name{Curriculum}{Vitae}
\title{Antoine Paret}
\address{8,rue de Delle}{90400, S\'evenans}{France}
\phone[mobile]{06~04~49~86~98}
\email{antoine.paret@utbm.fr}

\begin{document}
\makecvtitle

\section{Formation/Etudes}
\cventry{2015}{Obtention du Baccalaur\'eat S\'erie S}{Lyc\'ee Du dauphin\'e}{}{\textit{mention Bien}}%
{Option S.I /Sp\'e I.S.N/ Mention Euro Allemand}
\cventry{2015/2016}{\'Ecole d'ing\'enieur}{Univerist\'e de Technologie de Belfort Montb\'eliard (UTBM)}{} {\textit{2 \`eme ann\'ee}}{Tronc Commun}

\section{Compétences}
\cvitemwithcomment{Langues}{Anglais B1. Allemand scolaire}{}
\cvitemwithcomment{Bureautique}{Maitrise des outils de bureautique (Excel, Powerpoint, Word, LaTeX)}{}
\cvitemwithcomment{C.A.O}{Maitrise du logiciel SolidWorks, Solid Dynamic System, ainsi que Catia v5}{}
\cvitemwithcomment{Languages de programmation}{codage en bash, Java, C, VBA, Arduino, HTML}{}


\section{Experience professionnelle}

\cventry{Janvier 2015- Septembre 2016}{Staff integration 2016}{}{}{}{%
    \begin{itemize}
        \item Membre de l’organisation de l’intégration des nouveaux étudiants arrivant a l’UTBM en septembre 2016.
            5 mois de préparations, 3 semaines d’activités.
            Gestion d’un budget, delegation des taches, entrevue de la gestion administrative d’évenements.;
\end{itemize}}

\cventry{Janvier 2016}{Stage chez Amazon.fr}{Montelimar}{}{}{%
    \begin{itemize}
        \item Dur\'ee : un mois;
        \item Service maintenance chez Amazon;
        \item Stage enrichissant --> Découverte du fonctionnement d’une grande entreprise;
\end{itemize}}

\cventry{Juillet 2015 et Juillet 2016}{Emploi saisonier}{Genissieux}{}{}{%
    \begin{itemize}
        \item Collecte d’abricots chez Philippe Mortas durant des periodes d'un mois, en horaires postés.:
\end{itemize}}

\cventry{Janvier-Avril 2015}{Réalisation d’un jeu complet en Java}{}{}{}{%
    \begin{itemize}
        \item Réalisation d’un jeu de plateforme en deux dimensions;
        \item Projet réalisé intégralement en binôme (création des images, création de la structure du programme, de chacune de ses méthodes…);
        \item  Jeu complet :avec tutoriel, éditeur de niveaux et plus d’une trentaine de niveaux jouables.;
\end{itemize}}

\cventry{Mars 2012}{Réalisation d’un ballon Stratosphérique}{}{}{}{%
    \begin{itemize}
        \item  Réalisation d’un ballon sonde transportant deux caméras, des capteurs de pression, température, humidité et de vitesse.
        \item Dans le cadre de la matière technologie en 3em, très forte implication dans ce projet.
\end{itemize}}

\section{Centres d'int\'erets}

\cventry{2014}{Administrateur Rivlink}{}{Sevenans}{}{%
    \begin{itemize}
        \item Membre actif de l’association RivLink, association de gestion du réseau dans une résidence de plus de 150 studios.
            Connaissances du fonctionnement de ce réseau.;
\end{itemize}}
\cventry{2013-2015}{Création d’un club sportif de parkour}{}{}{}{%
    \begin{itemize}
        \item Création et rattachement à un club de Gymnastique du groupe de Parkour « Air Tribute » ainsi que création
            d’une page Facebook recensant toute les activités du club.
\end{itemize}}
\cventry{2014}{Actualité scientifique}{}{}{}{%
    \begin{itemize}
        \item Intérêt pour les nouvelles technologies. Lecture de magazines scientifiques tels que « science » et
            visionnage régulier de vidéos scientifiques (TEDx...).;
\end{itemize}}

\end{document}
